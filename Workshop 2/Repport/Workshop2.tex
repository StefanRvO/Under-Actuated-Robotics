\documentclass[a4paper,12pt,english]{article}
\usepackage[a4paper]{geometry}
\usepackage[english]{babel}
\usepackage{verbatim}
\usepackage{minted}
\usepackage[utf8]{inputenc}
\usepackage{graphicx}
\usepackage{float}
\usepackage{caption}
\usepackage{ mathrsfs }
 \usepackage{amsmath} \title{\textbf{Under Actuated Robotics - Workshop 2}}
\author{Stefan Ravn van Overeem – stvan13@student.sdu.dk}

\begin{document}
\maketitle
\newpage

\section{Derivation of State Space Representation}
The equations of motion for the pendulum can be expressed as:
$$m \cdot  l^2  \cdot \ddot{\theta} + b  \cdot \dot{\theta} + m  \cdot g  \cdot l  \cdot sin(\theta) = u$$

We assume small angle, and thus get
$$m \cdot  l^2  \cdot \ddot{\theta} + b  \cdot \dot{\theta} + m  \cdot g  \cdot l  \cdot \theta = u$$


We write $\ddot{\theta}$ as $\dot{x_2}$,
$\dot{\theta}$ as $x_2$
and $\theta$ as $x_1$

Thus we can write

$$\dot{x_1} = x_2$$
$$\dot{x_2} = -\frac{g}{l} \cdot x_1 - \frac{b}{m \cdot l^2} \cdot x_2 + \frac{u}{m \cdot l^2}$$

We can thus write the state space equation
$$\dot{x} = A \cdot x + B \cdot u$$

$$
\begin{bmatrix}
\dot{x_1} \\
\dot{x_2}
\end{bmatrix}
=
\begin{bmatrix}
	0 & 1 \\
	-\frac{g}{l} & -\frac{b}{m \cdot l^2}
\end{bmatrix}
\cdot
\begin{bmatrix}
x_1 \\
x_2
\end{bmatrix}
+
\begin{bmatrix}
0 \\
\frac{1}{m \cdot l^2}
\end{bmatrix}
\cdot u
$$

Thus we get
$
A = \begin{bmatrix}
	0 & 1 \\
	-\frac{g}{l} & -\frac{b}{m \cdot l^2}
\end{bmatrix}
$
and
$
B = \begin{bmatrix}
0 \\
\frac{1}{m \cdot l^2}
\end{bmatrix}
$

To get the output to equal our state, we set the C matrix to the identity matrix, and the D matrix to 0.


\end{document}
